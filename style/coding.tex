%%%% 设置listings宏包用来粘贴源代码
%% 方便粘贴源代码,部分代码高亮功能
\usepackage{listings}
\usepackage{color}

\DeclareCaptionFont{red}{\color{red}}

%% 所要粘贴代码的编程语言
\lstloadlanguages{{[LaTeX]TeX}, {[ISO]C++}, {Java}, {Ruby}, {Python}, {Scala}}

% Some useful colors...
\definecolor{darkviolet}{rgb}{0.5,0,0.4}
\definecolor{darkgreen}{rgb}{0,0.4,0.2} 
\definecolor{darkblue}{rgb}{0.1,0.1,0.9}
\definecolor{darkgrey}{rgb}{0.5,0.5,0.5}
\definecolor{lightblue}{rgb}{0.4,0.4,1}

\definecolor{stringColor}{rgb}{0.16,0.00,1.00}
\definecolor{annotationColor}{rgb}{0.39,0.39,0.39}
\definecolor{keywordColor}{rgb}{0.50,0.00,0.33}
\definecolor{commentColor}{rgb}{0.25,0.50,0.37}
\definecolor{javadocColor}{rgb}{0.25,0.37,0.75}
\definecolor{jTagColor}{rgb}{0.50,0.62,0.75}
\definecolor{eTagColor}{rgb}{0.50,0.62,0.75}
\definecolor{lineNumberColor}{rgb}{0.47,0.47,0.47}

% Example:
% \lstdefinestyle{SQL}{
%     language={SQL},basicstyle=\ttfamily, 
%     moredelim=**[is][\btHL]{`}{`},
%     moredelim=**[is][{\btHL[fill=orange!30,draw=red,dashed,thin]}]{@}{@},
% }
% 
% A listing with {\btHL highlighting of all \textbf{important} elements} looks as follows:
% 
% \begin{lstlisting}[style=SQL]
% SELECT name, password `FROM` users @WHERE@ name=@UNION SELECT@
% \end{lstlisting}
%
\makeatletter
\newenvironment{btHighlight}[1][]
{\begingroup\tikzset{bt@Highlight@par/.style={#1}}\begin{lrbox}{\@tempboxa}}
{\end{lrbox}\bt@HL@box[bt@Highlight@par]{\@tempboxa}\endgroup}

\newcommand\btHL[1][]{%
  \begin{btHighlight}[#1]\bgroup\aftergroup\bt@HL@endenv%
}
\def\bt@HL@endenv{%
  \end{btHighlight}%   
  \egroup
}
\newcommand{\bt@HL@box}[2][]{%
  \tikz[#1]{%
    \pgfpathrectangle{\pgfpoint{1pt}{0pt}}{\pgfpoint{\wd #2}{\ht #2}}%
    \pgfusepath{use as bounding box}%
    \node[anchor=base west, fill=green!30,outer sep=0pt,inner xsep=0pt, inner ysep=0pt, rounded corners=0.5pt, minimum height=\ht\strutbox+1.1pt,#1]{\raisebox{1.1pt}{\strut}\strut\usebox{#2}};
  }%
}
\makeatother

%% 设置listings宏包的一些全局样式
%% 参考http://hi.baidu.com/shawpinlee/blog/item/9ec431cbae28e41cbe09e6e4.html
\lstset{
numberbychapter=true,
breakatwhitespace=true,
showstringspaces=false,              %% 设定是否显示代码之间的空格符号
basicstyle=\scriptsize\ttfamily,           %% 设定字体大小\tiny, \scriptsize, \footnotesize, \small, \Large等等
keywordstyle=\color{keywordColor}\bfseries,
commentstyle=\color{red!50!green!50!blue!50},      
stringstyle=\color{stringColor},  
escapechar=`,                        %% 中文逃逸字符,用于中英混排
xleftmargin=1.5em,xrightmargin=0em, aboveskip=1.0em,
breaklines,                          %% 这条命令可以让LaTeX自动将长的代码行换行排版
extendedchars=false,                 %% 这一条命令可以解决代码跨页时,章节标题,页眉等汉字不显示的问题
frameround=fttt,
captionpos=top,
belowcaptionskip=1em
}

\lstdefinestyle{numbers}{
   numbers=left,
   numberstyle=\tiny\color{lineNumberColor}\lstfontfamily,
   stepnumber=1,
   numbersep=1em,
}

\lstdefinestyle{C++}{
   language=C++,
   texcl=true,
   prebreak=\textbackslash,
   breakindent=1em,
   emphstyle=\bfseries,
   keywordstyle=\color{keywordColor}\bfseries, %% 关键字高亮
   stringstyle=\color{stringColor},  
   morekeywords={alignas, alignof, char16_t, char32_t, constexpr, decltype, noexcept, nullptr, static_assert, thread_local, override, final},
   moredelim=**[is][\btHL]{@}{@},
   %style=numbers,
   %frame=leftline,                     %% 给代码加框
   %framerule=2pt,
   %rulesep=5pt
}

\lstnewenvironment{minicpp}[1][]
  {\setstretch{1}
  \lstset{style=C++, xleftmargin=0.0em, #1}}
  {}

\lstnewenvironment{c++}[1][]
  {\setstretch{1}
  \lstset{style=C++, #1}}
  {}


%\captionsetup[lstlisting]{textfont=red}
%{labelfont=bf, singlelinecheck=off, labelsep=space, textfont=red}

\lstdefinestyle{Java}{
   language=Java,
   texcl=true,
   prebreak=\textbackslash,
   breakindent=1em,
   keywordstyle=\bfseries, %% 关键字高亮
   morekeywords={}
   style=numbers,
   %frame=leftline,                     %% 给代码加框
   %framerule=2pt,
   %rulesep=5pt
}

\lstnewenvironment{java}[1][]
  {\setstretch{1}
  \lstset{style=Java, #1}}
  {}

\lstdefinestyle{Ruby}{
   language=Java,
   texcl=true,
   prebreak=\textbackslash,
   breakindent=1em,
   keywordstyle=\bfseries, %% 关键字高亮
   morekeywords={}
   style=numbers,
   %frame=leftline,                     %% 给代码加框
   %framerule=2pt,
   %rulesep=5pt
}

\lstnewenvironment{ruby}[1][]
  {\setstretch{1}
  \lstset{style=Ruby, #1}}
  {}

\lstdefinestyle{Python}{
   language=Python,
   texcl=true,
   prebreak=\textbackslash,
   breakindent=1em,
   emphstyle=\bfseries,
   keywordstyle=\color{keywordColor}\bfseries, %% 关键字高亮
   stringstyle=\color{stringColor},  
   morekeywords={with, as},
   moredelim=**[is][\btHL]{@}{@},
   %frame=leftline,                     %% 给代码加框
   %framerule=2pt,
   %rulesep=5pt
}

\lstnewenvironment{python}[1][]
  {\setstretch{1}
  \lstset{style=Python, #1}}
  {}

\lstdefinestyle{Scala}{
   language=Scala,
   texcl=true,
   prebreak=\textbackslash,
   breakindent=1em,
   keywordstyle=\bfseries, %% 关键字高亮
   morekeywords={}
   style=numbers,
   %frame=leftline,                     %% 给代码加框
   %framerule=2pt,
   %rulesep=5pt
}

\lstnewenvironment{scala}[1][]
  {\setstretch{1}
  \lstset{style=Scala, #1}}
  {}  

\renewcommand{\lstlistingname}{示例代码}
\renewcommand\thefigure{\thechapter-\arabic{figure}}

\newcommand\refcode[1]{{\itshape \lstlistingname\ascii{\ref{code:#1}(第\pageref{code:#1}页)}}}

